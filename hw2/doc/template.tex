\documentclass[12pt]{article}
\usepackage[utf8]{inputenc}
\usepackage[dvips]{graphicx}
\usepackage{epsfig}
\usepackage{fancybox}
\usepackage{verbatim}
\usepackage{array}
\usepackage{latexsym}
\usepackage{alltt}
\usepackage{amssymb}
\usepackage{amsmath}
\usepackage{hyperref}
\usepackage{listings}
\usepackage{color}
\usepackage{algorithm}
\usepackage{algpseudocode}
\usepackage[hmargin=3cm,vmargin=5.0cm]{geometry}
\usepackage{epstopdf}
\topmargin=-1.8cm
\addtolength{\textheight}{6.5cm}
\addtolength{\textwidth}{2.0cm}
\setlength{\oddsidemargin}{0.0cm}
\setlength{\evensidemargin}{0.0cm}
\newcommand{\HRule}{\rule{\linewidth}{1mm}}
\newcommand{\kutu}[2]{\framebox[#1mm]{\rule[-2mm]{0mm}{#2mm}}}
\newcommand{\gap}{ \\[1mm] }
\newcommand{\Q}{\raisebox{1.7pt}{$\scriptstyle\bigcirc$}}
\newcommand{\minus}{\scalebox{0.35}[1.0]{$-$}}



\lstset{
    %backgroundcolor=\color{lbcolor},
    tabsize=2,
    language=MATLAB,
    basicstyle=\footnotesize,
    numberstyle=\footnotesize,
    aboveskip={0.0\baselineskip},
    belowskip={0.0\baselineskip},
    columns=fixed,
    showstringspaces=false,
    breaklines=true,
    prebreak=\raisebox{0ex}[0ex][0ex]{\ensuremath{\hookleftarrow}},
    %frame=single,
    showtabs=false,
    showspaces=false,
    showstringspaces=false,
    identifierstyle=\ttfamily,
    keywordstyle=\color[rgb]{0,0,1},
    commentstyle=\color[rgb]{0.133,0.545,0.133},
    stringstyle=\color[rgb]{0.627,0.126,0.941},
}


\begin{document}

\noindent
\HRule %\\[3mm]
\small
\begin{center}
	\LARGE \textbf{CENG 483} \\[4mm]
	\Large Introduction to Computer Vision \\[4mm]
	\normalsize Spring 2018-2019 \\
	\Large Take Home Exam 2 \\
	\Large Object Recognition \\
    \Large Student Random ID: \\
\end{center}
\HRule

\begin{center}
\end{center}
\vspace{-10mm}
\noindent\\ \\ 
Please fill in the sections below only with the requested information. If you have additional things to mention, you can use the last section. Please note that all of the results in this report should be given for the \textbf{validation set}. Also, when you are expected to comment on the effect of a parameter, please make sure to fix other parameters.

\section{Local Features (15 pts)}

    \begin{itemize}
        \item Explain SIFT and Dense-SIFT in your own words. What is the main difference?
        
        \item Put your quantitative results (classification accuracy) regarding several values of SIFT and Dense-SIFT parameters here. Discuss the effect of these parameters.
    \end{itemize}


\section{Bag of Features (35 pts)}
    \begin{itemize}
        \item How did you implement BoF? Briefly explain.
        \item Give pseudo-code for obtaining the dictionary.
        \item Give pseudo-code for obtaining BoF representation of an image once the dictionary is formed. 
        \item Put your quantitative results (classification accuracy) regarding several values of BoF parameters here (e.g. $k$ of k-means algorithm). Discuss the effect of these.
    \end{itemize}


\section{Classification (20 pts)}
    \begin{itemize}
        \item Put your quantitative results regarding k-Nearest Neighbor Classifier parameters here. Discuss the effect of these briefly.
        \item What is accuracy and how do you evaluate it? Briefly explain.
        \item Give confusion matrices for classification results of several combinations of your choice. For example, you may try different $k$ values (of k-means and k-nearest neighbor, seperately, of course) or local features.
    \end{itemize}

\section{Your Best Configuration (30 pts)}
    \begin{itemize}
        \item You may try different combinations by changing parameters above. Simply give your best accuracy for the validation set. How did you decide to use this configuration?
        
        
        \item Explain your setup for this best accuracy. How can we reproduce your result using your code?
        
        \item Visualize confusion matrix for your best classification and briefly interpret the values.
        
    \end{itemize}

\section{Additional Comments and References}

    (if there any)





\end{document}

